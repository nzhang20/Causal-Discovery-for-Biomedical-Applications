% !TEX TS-program = xelatex
% !BIB TS-program = bibtex
\documentclass[12pt,letterpaper]{article}
\usepackage{style/dsc180reportstyle} % import dsc180reportstyle.sty
\usepackage{wasysym}
\newcommand{\ind}{\perp\!\!\!\!\perp}
\SetKwComment{Comment}{/* }{ */}

\def\multimapboth{\kern2pt\hbox{$\circ$}\kern-1pt\hbox{$-$}\kern-1pt\hbox{$\circ$}\kern2pt}
\def\circrightarrow{\kern2pt\hbox{$\circ$}\kern-1.5pt\hbox{$\to$}\kern2pt}
\def\circleftarrow{\kern2pt\hbox{$\leftarrow$}\kern-1.5pt\hbox{$\circ$}\kern2pt}

%%%%%%%%%%%%%%%%%%%%%%%%%%%%%%%%%%%%%%%%%%%%%%%%%%%%%%%%
%%%% Title and Authors
%%%%%%%%%%%%%%%%%%%%%%%%%%%%%%%%%%%%%%%%%%%%%%%%%%%%%%%%

\title{Causal Discovery in Gut Microbes for PCOS}

\author{Mariana Paco Mendivil \\
  {\tt mpacomendivil@ucsd.edu} \\\And
  Candus Shi \\
  {\tt c6shi@ucsd.edu} \\\And
  Nicole Zhang \\
  {\tt nwzhang@ucsd.edu} \\\And
  Biwei Huang \\
  {\tt bih007@ucsd.edu} \\\And
  Jelena Bradic \\
  {\tt jbradic@ucsd.edu}}

\begin{document}
\maketitle

%%%%%%%%%%%%%%%%%%%%%%%%%%%%%%%%%%%%%%%%%%%%%%%%%%%%%%%%
%%%% Abstract and Links
%%%%%%%%%%%%%%%%%%%%%%%%%%%%%%%%%%%%%%%%%%%%%%%%%%%%%%%%


\begin{abstract}
    \textcolor{black}{
    The human gut microbiome has become a significant factor in the understanding of metabolic health...}
\end{abstract}

\begin{center}

Code: \url{https://github.com/nzhang20/Causal-Discovery-for-Biomedical-Applications}
\end{center}

\maketoc
\clearpage

%%%%%%%%%%%%%%%%%%%%%%%%%%%%%%%%%%%%%%%%%%%%%%%%%%%%%%%%
%%%% Main Contents
%%%%%%%%%%%%%%%%%%%%%%%%%%%%%%%%%%%%%%%%%%%%%%%%%%%%%%%%

\section{Abstract}
The human gut microbiome has become a significant factor in understanding metabolic health, influencing conditions such as type 2 diabetes (T2D) and polycystic ovary syndrome (PCOS). Despite its recognized impact, much of the current research on the human gut microbiome and T2D remains limited to associative studies, leaving gaps in understanding the underlying causal relationships. This study addresses these gaps by enhancing causal discovery algorithms and integrating them with predictive modeling to investigate microbial contributions to T2D and PCOS.

Using cross-sectional datasets, including the Stanford T2D Gut Microbiome dataset and a PCOS Lancet dataset, we apply …blah blah blah…

Our methodology incorporates … blah blah blah ….

Our results reveal distinct microbial patterns associated with disease states and show the importance of causal discovery for understanding microbiome-disease interactions. This work provides a foundation for further research on treatment strategies influenced by the microbiome.


\section{Introduction}
The human gut microbiome has gained significant attention in recent years for its important role in metabolic health. While there has been extensive research that links the microbiome to health disorders such as type 2 diabetes (T2D) and polycystic ovary syndrome (PCOS), the majority of these studies remain correlational, leaving causal relationships undiscovered. Understanding these relationships is essential to improving and personalizing medical treatments for such diseases.

This study builds on recent advancements in causal discovery algorithms to investigate how microbial taxa influence metabolic disorders. Our goal is to find patterns that conventional association-based methods might miss by focusing on datasets with extensive metadata and utilizing tools made for high-dimensional, clustered data. Our work emphasizes the importance of addressing data complexities, such as clustering from varied study designs, to ensure robust causal inference.

[briefly talk about the methods we will use]


\subsection{Literature Review}

\subsection{Data}
To answer our research question, we used data from an individual participant data (IPD) meta analysis and systematic review conducted by \citep{yang2024pcos}. This means the dataset is an aggregation of the 14 studies that were included in the systematic review, but at the individual level. This is different from a meta analysis which analyzes aggregated data or statistics from multiple different studies. Each row of this PCOS dataset represents one sample of gut microbe abundance measurements as well as the sample's study's region, the sample's classification as a PCOS patient or a healthy control (HC), and if they were a PCOS patient, whether they had low (LT) or high (HT) testosterone levels. This granularity gives us more data and statistical power behind our results rather than using just one PCOS study. Although IPD meta analyses have these advantages as well as advantages over regular aggregated meta analyses, there are some glaring issues that must be addressed before we work with this dataset as is. In particular, the dataset is clustered because it consists of participants from different studies where each study had their own unique recruitment and sampling methods. Thus, we must account for clustering while running all analyses on this large aggregated dataset \citep{riley2010ipdma}. Two ways to do this in a typical meta analysis is via a fixed-effect model or a random-effects model \citep{dettori2022fixedrandomeffect}. The choice of one over the other depends on the situation at hand, and since our PCOS dataset is comprised of 11 studies from China, 1 study from Poland, 1 study from Austria, and 1 study from Russia, the appropriate model appears to be a random effects model as there are many other areas of the world that we are interested in. (Not sure if any of this makes sense in the realm of causal discovery, but will leave here for now). (Alternatively) However, if we take a look at the scatter plots between pairs of microbes, there are no clear clusters representing different study regions or the individual studies. (Perhaps we can't see clusters because it's in such high dimension. We must run some clustering method to see if there is any clustering. Are the PCOS patients from different studies more alike or are all patients from one study whether PCOS or HC more alike than a different study?)

\section{Methods}

\section{Results}

\section{Discussion}

\section{Conclusion}

%%%%%%%%%%%%%%%%%%%%%%%%%%%%%%%%%%%%%%%%%%%%%%%%%%%%%%%%
%%%% Reference / Bibliography
%%%%%%%%%%%%%%%%%%%%%%%%%%%%%%%%%%%%%%%%%%%%%%%%%%%%%%%%

\clearpage
\makereference

\bibliography{reference}
\bibliographystyle{style/dsc180bibstyle}

%%%%%%%%%%%%%%%%%%%%%%%%%%%%%%%%%%%%%%%%%%%%%%%%%%%%%%%%
%%%% Appendix
%%%%%%%%%%%%%%%%%%%%%%%%%%%%%%%%%%%%%%%%%%%%%%%%%%%%%%%%

\clearpage
\makeappendix

\end{document}