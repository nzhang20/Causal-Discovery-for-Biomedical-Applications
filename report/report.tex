% !TEX TS-program = xelatex
% !BIB TS-program = bibtex
\documentclass[12pt,letterpaper]{article}
\usepackage{style/dsc180reportstyle} % import dsc180reportstyle.sty
\usepackage{wasysym}
\newcommand{\ind}{\perp\!\!\!\!\perp}
\SetKwComment{Comment}{/* }{ */}

\def\multimapboth{\kern2pt\hbox{$\circ$}\kern-1pt\hbox{$-$}\kern-1pt\hbox{$\circ$}\kern2pt}
\def\circrightarrow{\kern2pt\hbox{$\circ$}\kern-1.5pt\hbox{$\to$}\kern2pt}
\def\circleftarrow{\kern2pt\hbox{$\leftarrow$}\kern-1.5pt\hbox{$\circ$}\kern2pt}

%%%%%%%%%%%%%%%%%%%%%%%%%%%%%%%%%%%%%%%%%%%%%%%%%%%%%%%%
%%%% Title and Authors
%%%%%%%%%%%%%%%%%%%%%%%%%%%%%%%%%%%%%%%%%%%%%%%%%%%%%%%%

\title{Causal Discovery in Gut Microbes for PCOS}

\author{Mariana Paco Mendivil \\
  {\tt mpacomendivil@ucsd.edu} \\\And
  Candus Shi \\
  {\tt c6shi@ucsd.edu} \\\And
  Nicole Zhang \\
  {\tt nwzhang@ucsd.edu} \\\And
  Biwei Huang \\
  {\tt bih007@ucsd.edu} \\\And
  Jelena Bradic \\
  {\tt jbradic@ucsd.edu}}

\begin{document}
\maketitle

%%%%%%%%%%%%%%%%%%%%%%%%%%%%%%%%%%%%%%%%%%%%%%%%%%%%%%%%
%%%% Abstract and Links
%%%%%%%%%%%%%%%%%%%%%%%%%%%%%%%%%%%%%%%%%%%%%%%%%%%%%%%%


\begin{abstract}
    \textcolor{black}{
    The human gut microbiome has become a significant factor in understanding metabolic health, influencing conditions such as type 2 diabetes (T2D) and polycystic ovary syndrome (PCOS). Despite its recognized impact, much of the current research on the human gut microbiome and diseases remain limited to associative and correlational studies, leaving gaps in understanding the underlying causal relationships. This study addresses these gaps by utilizing causal discovery algorithms and causal inference methods and comparing them with prediction models to investigate microbial contributions to T2D and PCOS. First, we graph the microbe-microbe interaction networks on the genus level for healthy and diseased cohorts using a version of the Peter-Clark (PC) algorithm altered to reduce the multiple testing burden. Then, we graph the microbe-disease interaction network on the genus level for a disease using the constraint-based causal discovery from heterogeneous/nonstationary data algorithm (CD-NOD) and compare the microbes directly linked to disease with microbes from a variational autoencoder (VAE) prediction model. Our results show that there are microbes causal to T2D and microbes causal to PCOS (expand). This work aims to provide a framework for investigating causal relationships between the gut microbiome and other diseases as well as guide further research and wet-lab experiments and develop a stronger understanding of the role of the gut microbiome in precision medicine.}
\end{abstract}

\begin{center}

Code: \url{https://github.com/nzhang20/Causal-Discovery-for-Biomedical-Applications}
\end{center}

\maketoc
\clearpage

%%%%%%%%%%%%%%%%%%%%%%%%%%%%%%%%%%%%%%%%%%%%%%%%%%%%%%%%
%%%% Main Contents
%%%%%%%%%%%%%%%%%%%%%%%%%%%%%%%%%%%%%%%%%%%%%%%%%%%%%%%%

\section{Introduction}

The human gut microbiome has gained significant attention in recent years for its important role in metabolic health. While there has been extensive research that links the microbiome to health disorders such as type 2 diabetes (T2D) \citep{zhou2019t2d} and polycystic ovary syndrome (PCOS) \citep{yang2024pcos}, the majority of these studies remain correlational, leaving causal relationships undiscovered. Understanding these relationships is essential to improving and personalizing medical treatments for such diseases.
This study builds on recent advancements in causal discovery algorithms to investigate how microbial taxa influence metabolic disorders. Our goal is to find patterns that conventional association-based methods might miss by leveraging the marginal and conditional independencies found in the data as well as network theory to assess where a causal relationship might occur and if possible, its causal direction. Given the high-dimensional nature of this type of data, we also explore different feature pruning techniques to reduce the multiple testing burden and for ease of interpretation. 

We focus on two aspects of causal discovery and causal inference in the gut microbiome. First, we are curious to see how the microbe-microbe interaction networks may differ between the two outcome groups. Due to the high number of features compared to the number of samples in gut microbiome abundance data, we first reduce the number of edges between microbes using a sparse correlation method, SparCC \citep{weiss2016correlationbenchmark, friedman2012sparcc}, and a sparse precision matrix estimator, graphical lasso \citep{friedman2007glasso}. Then, we graph the two networks for the corresponding cohorts using a constraint-based causal discovery algorithm similar to the Peter-Clark (PC) algorithm \citep{glymour2019review}, but with a smaller depth and without a direction orientation step to reduce the multiple testing burden.

Second, we are also interested in graphing the microbe-disease interaction network, where we are particularly interested in the microbes directly linked to disease status. Instead of reducing the number of edges, we reduce the number of features using logistic lasso regression to account for the outcome variable. Finally, we graph the network using the features that survive lasso and a disease status node using the constraint-based causal discovery from heterogeneous/nonstationary data algorithm (CD-NOD) \citep{huang2019cdnod} to identify microbes directly linked to disease status. Given the predictive nature of graphing a microbe-disease interaction network, we are also interested in developing a prediction model for disease using a causal representation learning technique xxx and comparing the results from CD-NOD to the microbes used in the model.

\subsection{Literature Review}

\textbf{T2D}

T2D is a metabolic disease where individuals have chronic high blood sugar, otherwise known as hyperglycaemia. This is a result of insulin resistance where the pancreas produces insulin, but the cells do not respond to it, leading the pancreas to try to produce more. The pancreas eventually fails to keep producing insulin leading to low insulin levels and high blood sugar, and this can lead to increased risks of developing other diseases such as heart disease and kidney disease \citep{adat2d}. T2D affects millions of people, and many studies have been conducted to investigate its underlying cause, its common precursor coined as “prediabetes”, and other factors that can contribute or affect the development of T2D \citep{tabak2012prediabetes, qin2012mgwast2d, mehta2000hcvt2d}. 

Given the impactful role of the gut microbiome on human health, numerous studies have also investigated the relationship between gut microbiota and T2D. For example, \citep{zhou2019t2d} conducted a longitudinal study of multi-omic data on healthy individuals vs individuals with prediabetes (an early stage of T2D) to determine how microbes behave differently between the two cohorts. They found that variation in microbes between and within individuals of each cohort differed, that each cohort responded to infections and immunizations differently, and through associations, that host-microbe interactions differed between the two cohorts. In particular, they found that “the genus \textit{Holdemania} was significantly associated with \textit{Clostridium XIVb} and \textit{Phascolarctobacterium} in insulin-sensitive participants, but significantly correlated with \textit{Clostridium XIVa}, \textit{Clostridium XVII}, \textit{Collinsella}, \textit{Lachnospiracea incertae sedis}, and \textit{unclassified Lachnospiraceae} in insulin-resistant participants”. \citep{baars2024gutt2d} also found common results from various studies investigating this relationship: there appears to be “a reduction of butyrate-producing bacteria such as \textit{Faecalibacterium}, \textit{Clostridium}, and \textit{Akkermansia} in individuals with T2D”. 

These analyses demonstrate that there are significant differences in microbe interactions between healthy and prediabetic individuals, and furthermore, that we can discover the causal graph from their data and use causality to determine which host-microbe interactions this study found through associations are not spurious, but causal. \newline

\textbf{PCOS}

PCOS is a complex endocrine disorder linked to metabolic diseases such as obesity and T2D. It affects 6-13\% of women of reproductive age, and 70\% of affected women remain undiagnosed as the causes of PCOS largely remains a mystery \citep{who2025pcos}. In fact, it was not until quite recently that the scientific community has peaked interest in studying PCOS and its causes. Current diagnostic methods use hormone and metabolic biomarkers, but these techniques are insufficient to differentiate between different PCOS subtypes, such as those characterized by hyperandrogenism. Due to inconsistent study findings, regional differences, and heterogeneity in studies, the association between PCOS and gut microbiota is not well-defined. \citep{yang2024pcos} conduct an individual participant data meta-analysis and systematic review to see if gut microbiota characteristics between healthy individuals and PCOS patients, between different subtypes of PCOS, and regional differences can be identified using data from a variety of clinical trials.

Using Wilcoxon tests with Benjamini-Hochberg corrected p-values, they found differential bacteria between the healthy and PCOS groups: PCOS patients had slightly lower levels of \textit{Bacillota} and higher levels of \textit{Actinobacteriota}; PCOS patients in China had lower alpha diversity than healthy controls, whereas PCOS patients in Europe had higher diversity; PCOS patients with high testosterone (HT) had different microbial patterns compared to those with low testosterone (LT), including lower levels of \textit{Faecalibacterium} and higher levels of \textit{Prevotella}. 

With biomarkers like \textit{Faecalibacterium} and \textit{Prevotella}, PCOS subtypes have distinct gut microbiota compositions that are impacted by geography and testosterone levels. These results highlight the possibility of personalized treatments based on microbiota. However, to handle population variety and improve strain-level assessments, extensive, global research is required. Given this complexity, we are interested in identifying potential biomarkers for all types of PCOS, i.e. disregarding the hyperandrogenism subtypes. \newline

\textbf{Causal Discovery and the Gut Microbiome}

There have been previous attempts to perform causal discovery on the gut microbiome. In particular, \citep{sazal2021causalgut} attempts to use causal discovery to construct causal networks and implement do-calculus, a causal inference technique developed by \citep{pearl2016primer} to estimate the causal effects of microbes on other microbes and on outcome variables. For the causal discovery task, they use the PC-stable algorithm \citep{colombo2014pcstable} which is a variation of PC that removes order-dependence during the estimation of the skeleton of the casual graph. The advantage of PC-stable over PC is that PC may output different results given the order of the conditional independence tests done. After finding the causal graphs, they used do-calculus to quantify the effects of each edge in the graphs which essentially uses the do-operator to intervene on the treatment node, remove all edges pointing towards said node, and to estimate the interventional expectation of the outcome node using a model appropriate for the given data structure like linear regression. They test their pipeline’s consistency using simulations and apply their pipeline to real dataset of healthy individuals, individuals with ulcerative colitis (UC), and individuals with Crohn’s disease (CD). They used bootstraps to compute confidence intervals for each edge and permutation tests to calculate p-values for the overall network and found bacteria beneficial to UC such as \textit{unclassified Oscillibacter}, \textit{Sutterella wadsworthensis}, and \textit{Bacteroides xylanisolvens}. However, they fail to account for multiple testing issues and covariates in their networks. Since we designed our study before finding this paper, we see a promising role of causal discovery and causal inference in gut microbial data for studying various human diseases. 

Additionally, there have been advancement to causal discovery algorithms since the development of the PC and PC-stable algorithms. For example, a variant of the PC algorithm, CD-NOD \citep{huang2019cdnod}, was developed specifically for heterogeneous data, where the heterogeneity of the observed data can help discover the causal structure given certain variables that can change the distribution of the data. This is particularly useful with gut microbiome data where a dataset may contain samples from different studies, hence providing a heterogeneous dataset where the study ID can change the data distribution. 

\subsection{Data}

To answer our research question, we used the NIH Human Microbiome Project (HMP2) dataset \citep{zhou2019t2d} for T2D and the aggregated dataset from an individual participant data (IPD) meta analysis and systematic review conducted by \citep{yang2024pcos} for PCOS. 

The HMP2 dataset \citep{zhou2019t2d} followed 106 participants for up to four years, collecting blood, stool, and nasal samples at every self-reported healthy visit and additional visits during periods of respiratory viral infection (RVI), influenza immunization, and other stresses such as antibiotic treatment. Since we are interested in the gut microbes, we look specifically at the visits where gut microbial taxa were profiled using 16S sequencing which provides normalized gut microbe abundance for taxa classified at 6 phyla, 28 classes, 12 orders, 21 families, and 45 genera. As the study authors illustrate, the gut microbiome can fluctuate with the presence of antibiotics and other stressor events such as illness, so we also only look at the visits that were classified as “Healthy”. For each individual, there is information about their race, sex, age, BMI, steady-state plasma glucose (SSPG), and insulin sensitivity classification. For 66 participants, their insulin sensitivity was assessed using an insulin suppression test measured by SSPG: 31 individuals were insulin-sensitive (IS: SSPG < 150 mg/dl), and 35 individuals were insulin-resistant (IR: SSPG $\geq$150 mg/dl). The remaining 40 individuals are classified as unknown due to medical contraindications leading to a lack of insulin suppression tests. Since the dataset is longitudinal but with very few time points per subject, we treated it as a cross-sectional dataset, leaving us with 153 and 178 samples for the IS and IR cohorts respectively.

The IPD meta analysis dataset \citep{yang2024pcos} is an aggregation of the 14 studies that were included in the systematic review, but at the individual level. This is different from a meta analysis which analyzes aggregated data or statistics from multiple different studies. Each row of this PCOS dataset represents one sample of gut microbe abundance measurements as well as the sample’s study’s region (Asia or Europe), the sample’s classification as a PCOS patient or a healthy control (HC), and if they were a PCOS patient, whether they had low (LT) or high (HT) testosterone levels. This granularity gives us more data and statistical power behind our results rather than using just one PCOS study. Since the only considerations for confounding their selection criteria specified were no drug interventions, there are other gut microbiome-related confounders that may be present in our data, such as diet, alcohol usage, stress, etc. We examined the study designs of the 14 included studies and found that they varied in external factors including diet, alcohol consumption, the use of antibiotics, and more. Although this is a limitation with the dataset, we chose to continue with this dataset due to its large sample size. This dataset provided us with 1,128 genera and 435 HC \& 513 PCOS individuals. 

\section{Methods}

In this study, we use causal discovery algorithms and compare them with predictive modeling to explore the causal relationships between the gut microbiome and two diseases: T2D and PCOS. We used datasets that were cross-sectional, meaning they provide a snapshot of the gut microbiome and disease status at a single point in time, which makes it challenging to determine whether changes in the microbiome cause the disease or are a result of it. Rather than recovering this information from experiments that can be expensive, we can use computational methods to discover causality to the best of the data’s ability. 

Our approach tackles the complexities of working with high-dimensional data (many microbial features) and relatively small sample sizes. We use feature selection and sure screening techniques to reduce the dimensions of these datasets, and we adjust existing causal discovery algorithms to reduce the multiple testing burden. The goal is to build a framework for understanding how gut microbes contribute to disease and to identify potential targets for personalized treatments.  

\subsection{Data Preprocessing}

For the T2D dataset we removed subjects with an unknown insulin resistance status and selecting only the “Healthy” sample visits. We extracted microbial abundance data at the genus level and converted the values to percentages. The dataset was then merged across subject, sample, and microbial abundance files, with categorical variables like disease status (IRIS), gender, and ethnicity encoded numerically.

For the PCOS dataset, we grouped any unclassified microbial data into a single category and numerically encoded binary variables such as region, and disease status. To account for differences in the study sites, we created a study site variable by manually comparing the study sample sizes and regions.

Based on the suggestions provided by \citep{weiss2016correlationbenchmark} on different correlation strategies to use for different structures of a gut microbe dataset, we filtered out rare operational taxonomy units (OTUs), using a rareness threshold of 1\%. This helped reduce features substantially for the PCOS dataset from 1,128 genera to 274 genera. 

\subsection{Feature Selection and Sure Screening}

Given the high-dimensional nature of the PCOS dataset, we experimented with different feature selection and sure screening methods to reduce the feature space before running causal discovery algorithms to reduce the multiple testing burden on the causal discovery algorithms. The two tasks at hand call for different methods. For the microbe-microbe interaction network, since the algorithms start with a complete graph, we used SparCC and graphical lasso separately, to reduce the number of edges between pairs of microbes and removed nodes that were disconnected from any other node. For the microbe-disease interaction network, we used logistic lasso regression to remove features that did not contribute to the prediction of disease status. \newline

\textbf{SparCC}

SparCC is a method developed by \citep{friedman2012sparcc} to estimate correlations from compositional data, which are data that contain relative values such that each row adds up to the same value. In the case of gut microbiome data, 16s sequencing data will provide estimates of the relative abundance of microbes within a sample, meaning each sample’s values adds up to 100\%. Compositional data can produce spurious correlations because for any sample, each relative value are dependent on the values of the other features. This means each pair of features will “tend to have negative correlation regardless of the true correlation” and are not representative of the underlying mechanisms and relationships of the microbiome. \citep{weiss2016correlationbenchmark} also demonstrate that standard correlation techniques like Spearman and Pearson’s correlations perform poorly on their own when applied to compositional data. They suggest that these two correlation metrics can be paired with other methods like random matrix theory (RMT) and SparCC to improve their accuracy. 

SparCC is a method that makes two assumptions: (i) the number of different components/OTUs is large, and (ii) the true correlation network is sparse. First, it takes the log-ratio transformation of two OTUs $$y_{ij} = \log \frac{x_i}{x_j} = \log x_i - \log x_j$$ where $x_i$ is the relative abundance of $\text{OTU}_i$, to compute correlations based on true abundances of OTUs (rather than the relative), to establish independence between $y_{ij}$ and which OTUs are included in the analysis, and to allow $y_{ij}$ to be any real number. Namely, SparCC can compute correlations based on the true abundances of OTUs by using the following result from \citep{aitchison1982compositional}, $$t_{ij} \coloneq \text{Var}\left(\log \frac{x_i}{x_j}\right) = \text{Var} (y_{ij})$$ where the variance is taken across all samples. A large $t_{ij}$ indicates there are samples with uncorrelated OTUs, and a $t_{ij} = 0$ means the OTUs are perfectly correlated. $t_{ij}$ can be written in terms of the true correlation: 
\begin{align*}
	t_{ij} &\coloneq \text{Var}\left(\log \frac{x_i}{x_j}\right) = \text{Var}\left(\log \frac{w_i}{w_j}\right) = \text{Var} (\log w_i - \log w_j) \\
	&= \text{Var} (\log w_i) + \text{Var} (\log w_j) - 2 \text{Cov}(\log w_i, \log w_j) \\
	&\coloneq \omega_i^2 + \omega_j^2 - 2 \rho_{ij}\omega_i\omega_j
\end{align*}
where $w_i, w_j$ are the true abundances of $\text{OTU}_i$ and $\text{OTU}_j$. Finally, given a sparse true correlation matrix, SparCC can approximate $\rho_{ij}$ as follows, $$\rho_{ij} = \frac{\omega_i^2 + \omega_j^2 - t_{ij}}{2\omega_i\omega_j}$$ Each of these components can be estimated via approximations outlined by \citep{friedman2012sparcc}, as the details are not relevant for our purpose. The important part of SparCC is that it uses an iterative procedure to estimate $\rho_{ij}$. Thus, the maximal number of iterations, the number of exclusion iterations, and the threshold can be specified.

We run SparCC in Python using the package: \url{https://github.com/dlegor/SparCC}, and the same parameters used by \citep{friedman2012sparcc, zhou2019t2d} of 20 iterations, 10 exclusion iterations, and a threshold of 0.1. P-values are obtained from 100 bootstraps. \newline

\textbf{Graphical Lasso}

An alternative to SparCC to reduce the edges in the microbe-microbe network is to apply the lasso penalty on the inverse covariance matrix. This method, graphical lasso, was developed by \citep{friedman2007glasso} and assumes that the data are multivariate normal with mean $\mu$ and a covariance matrix $\Sigma$. The inverse covariance matrix, $\Theta \coloneq \Sigma^{-1}$ is also known as the precision matrix where if $\Sigma_{ij}^{-1} = 0$, then variables $i$ and $j$ are conditionally independent given all of the other variables. The lasso component comes in when each variable is modeled by all other variables as predictors and applies the lasso penalty to obtain the coefficients of the predictors. Then, each row of $\Theta$ can be filled in by the covariates of this lasso model for each variable. 

This estimand is not novel, but \citep{friedman2007glasso} propose that their graphical lasso algorithm can estimate the precision matrix in a more simple and fast way than previous algorithms using pathwise coordinate descent. Again, the exact details are not relevant to our project, but it is important to highlight the distributional assumption of multivariate normality. This is most often not the case for gut microbiome data and may be assessed by checking the normality of the marginal distributions through qqplots. If the data do not satisfy this assumption for the precision matrix, (Gaussian represents second-order relationships, which is a pairwise Markov graph…).

Graphical lasso is implemented in R using the glasso package with a regularization parameter of 2 to reduce runtime. Graphical lasso with grid search on the regularization parameter can also be implemented to find a more optimal value. \newline

\textbf{Logistic Lasso Regression}

This is simply a logistic regression model penalized with the lasso penalty (the $\ell$-1 norm). We use a logistic regression model because the outcome variable of interest is disease status which is a binary variable. K-fold cross-validation logistic lasso regression is implemented in R using the glmnet package with the cv.glmnet function, 10 folds, and alpha = 1 for the lasso penalty.

\subsection{Causal Discovery Algorithms}

After removing edges and features, we proceed with the causal discovery algorithms. For the microbe-microbe interaction network, we perform a series of conditional independence tests for all pairs of microbes that have an edge between them, conditioned on sets of size 1 and 2. Then, we orient the edges as much as possible using Meek’s rules. For the microbe-disease interaction network, we apply CD-NOD using the study site and region as the heterogeneity index. \newline

\textbf{PC algorithm}

In order to introduce our algorithm, we first must explain the PC algorithm. PC is one of the oldest and widely-used general-purpose causal discovery algorithms in the current literature \citep{glymour2019review}. At a high level, PC is a constraint-based search algorithm that starts with a complete graph, and constrained by the unconditional and conditional independencies found in the data, removes edges between two variables. Then, PC will orient as many of the edges as it can based on preserving v-structures and Meek’s rules based on directed graph theory. PC may not be able to orient all of the edges, leaving some undirected edges. This sort of output is known as a completed partially directed acyclic graph (CPDAG) which is a DAG with a mixture of directed and undirected edges. The CPDAG is a representation of the Markov Equivalence Class (MEC), a collection of all DAGs that are Markov equivalent, i.e. graphs with the same d-separation properties and implying the same conditional independence relations. 

Briefly, the PC steps are: 
\begin{enumerate}
    \item Start with a complete undirected graph
    \item Causal skeleton discovery
    \item Find v-structures
    \item Orientation propagation via Meek rules
\end{enumerate}

A more detailed algorithm is outlined in the Appendix \citep{spirtes2000causation}. 

PC assumes iid data for consistency, no latent confounders, the Causal Markov condition, and the Faithfulness assumption. 

\textit{Causal Markov condition}. Every variable $X$ in the set of variables $\textbf{V}$ is independent of its non-descendants given its parents. 

\textit{Faithfulness assumption}. The only independencies among the variables $\textbf{V}$ are those entailed by the Causal Markov Condition.

The Causal Markov condition and Faithfulness assumption together give us necessary and sufficient conditions for learning the causal graph from conditional independencies. 

PC works with all data types as long as the conditional independence tests used are appropriate for the empirical distribution of the data. For example, our dataset includes all continuous variables (normalized abundances of gut microbes), but our EDA shows us that they are not linear nor Gaussian. Thus, we ought to use non-parametric conditional independence tests, such as KCI. However, non-parametric estimators do not perform well in high-dimensions with low sample sizes. Due to this tradeoff between distribution assumptions and statistical power limitations, we must carefully consider whether to use a linear parametric test like Fisher-Z or a nonparametric test like KCI. But, in more general cases, PC’s greatest limitation is arguably the assumption that there are no latent confounders. \newline

\textbf{Our algorithm}

The main issue with the PC is the series of conditional independence tests conducted on a fixed threshold of $\alpha = 0.05$. Due to the number of features in our dataset, the number of conditional independence tests conducted can be quite large and also impact algorithm complexity. This brings into the conversation a multiple-testing issue that is not being corrected. We attempt to minimize the prevalence of this issue with our own variation on the constraint-based search algorithm. To correct for this statistical shortcoming, we reduce the number of tests done by taking advantage of the correlational findings from the preceding feature reduction and sure screening step. The remaining steps of PC regarding direction orientation remain the same. 

One pitfall of our algorithm is that it assumes that the correlations found using SparCC or graphical lasso are a superset of the set of all causal relations. This may not be the case due to a well-known phenomenon called Simpson’s paradox, which essentially demonstrates that a statistical association in the data for an entire population may be reversed in every sub-population, e.g. when new information or variable is conditioned for \citep{pearl2016primer}. In other words, there may be certain causal relationships that are not statistically correlated due to a lack of information. However, this is not so far-fetched as the other well-established algorithms we use assume Faithfulness and solely rely on conditional independencies found in the data (and d-separation rules) to identify all casual relations. \newline

\textbf{CD-NOD}

\subsection{Variational Autoencoder}

\section{Results}

\subsection{EDA}

\subsection{Microbe-Microbe Interaction Network}

\subsection{Microbe-Disease Interaction Network}

\subsection{VAE Model}

\section{Discussion}

\section{Conclusion}

%%%%%%%%%%%%%%%%%%%%%%%%%%%%%%%%%%%%%%%%%%%%%%%%%%%%%%%%
%%%% Reference / Bibliography
%%%%%%%%%%%%%%%%%%%%%%%%%%%%%%%%%%%%%%%%%%%%%%%%%%%%%%%%

\clearpage
\makereference

\bibliography{reference}
\bibliographystyle{style/dsc180bibstyle}

%%%%%%%%%%%%%%%%%%%%%%%%%%%%%%%%%%%%%%%%%%%%%%%%%%%%%%%%
%%%% Appendix
%%%%%%%%%%%%%%%%%%%%%%%%%%%%%%%%%%%%%%%%%%%%%%%%%%%%%%%%

\clearpage
\makeappendix

Please see a copy of our \href{https://drive.google.com/file/d/1fEGpmoSULPbpG6bcopPBKdlyF6gG_Gv_/view?usp=sharing}{project proposal}. 

\subsection{Algorithms}

\begin{algorithm}[H]
\caption{PC}
\label{alg:pc}
\DontPrintSemicolon
\LinesNumbered
\KwIn{Data, $\mathbf D$}
\KwOut{CPDAG, $G$}
$G \gets \text{the complete undirected graph over $\mathbf D$}$\;
$\textbf{Sepset}(A, B) = \textbf{Sepset}(B, A) \gets \emptyset \quad \text{(the d-separation set of $A$ and $B$ $\forall A, B \in G$)}$\;
\tcp{Causal skeleton discovery}
\For{$(A, B)$ adjacent pairs in $G$} {
    $n \gets 0$\;
    \While{$\exists$ a set $C$ in $G$ adjacent to $A$ or $B$ s.t. $|C| = n$}{
        \If{$A \ind B | C$} {
            $\text{remove the edge between $A$ and $B$}$\;
            $\text{record $C$ in \textbf{Sepset}$(A, B)$ and \textbf{Sepset}$(B, A)$}$\;
            \textbf{break}
        }
        $n \gets n + 1$\;
    }
}
\tcp{Find v-structures}
\For{$(A, B, C)$ triple in $G$ s.t. $A - B - C$}{
    \If{$B \notin \textbf{Sepset}(A, C)$}{
        $\text{orient the edges $A \to B \leftarrow C$}$\;
    }
}
\tcp{Orientation propagation via Meek rules}
\While{there are edges to orient} {
    \If{$A \to B$, and $B$ and $C$ are adjacent, and $A$ and $C$ are not adjacent, and there is no arrowhead at $B$} {
        $\text{orient the edge $B - C$ as $B \to C$}$\;
    }
    \If{$\exists$ a directed path from $A$ to $B$, and an edge between $A$ and $B$}{
        $\text{orient the edge $A - B$ as $A \to B$}$\;
    }
}
\end{algorithm}

\begin{algorithm}
\caption{Our Algorithm}
\label{alg:ouralg}
\DontPrintSemicolon
\LinesNumbered
\KwIn{initial adjacency matrix, $A$}
\KwOut{updated adjacency matrix, $A$}
\tcp{Remove conditional independencies of conditioning set size 1}
\For{$(A_i, A_j)$ pairs where $A_{ij} = 1$} {
    \For{$A_k \neq A_i, A_j$} {
        \If{$A_i \ind A_j | A_k$} {
            $A_{ij}, A_{ji} \gets 0$\;
        }
    }
}
\tcp{Remove conditional independencies of conditioning set size 2}
\For{$(A_i, A_j)$ pairs where $A_{ij} = 1$} {
    \For{$A_k, A_l \neq A_i, A_j$ and $A_k \neq A_l$} {
        \If{$A_i \ind A_j | \{A_k, A_l\}$} {
            $A_{ij}, A_{ji} \gets 0$\;
        }
    }
}
\end{algorithm}

\clearpage
\textbf{Contributions}

MPM, CS, and NZ designed the project. MPM and CS found datasets. CS and NZ performed EDA. CS built the microbe-microbe and microbe-disease interaction networks. MPM built the VAE model. BH proposed the causal discovery algorithm for the microbe-microbe interaction network. JB proposed the feature reduction and sure screening methods. BH and JB provided insightful comments and suggestions to the design. MPM, CS, and NZ interpreted the results and wrote the final report.

\end{document}